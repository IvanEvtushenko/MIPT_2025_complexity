\section{Введение}

Задача о нахождении дерева Штейнера является одной из базовых задач комбинаторной оптимизации. Она впервые была предложена Якобом Штайнером в контексте геометрии, в последствии была обобщена на графы. Задача заключается в поиске дерева минимального веса во взвешенном графе, соединяющего некоторое заранее известное подмножество вершин, быть может, через дополнительные вершины исходного графа.

Эта задача часто возникает при решении задач оптимизации или построения карт телекоммуникационных и транспортных сетей.

В настоящей работе было рассмотрено доказательство $\mathbf{NP}$-полноты задачи и сведение её к метрической версии, представлено приближенное решение обоих. Проведён анализ на основе датасета I080.