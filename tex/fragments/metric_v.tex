\section{Сведение к метрической версии}

\subsection{Построение метрической версии}

Исходно имеем взвешенный граф $G = (V, E)$, с весовой функцией $\omega: E \rightarrow \mathbb{R}^{+}$, $\{v \in V_0\}$ - терминальные вершины.

\begin{definition}
    Получим из него метрический граф: пусть $G' = (V', E')$ - полный граф на множестве вершин $V$ ($V':=V)$, введём на нём следующую весовую функцию $d$: $E' \rightarrow  \mathbb{R}^{+}$, т.ч.
    \[
    d(u, v) = \min_{l:\ u \rightarrow v}\Bigl(\sum_{e \in l}(\omega(e))\Bigr),
    \]
    иначе говоря, вес каждого ребра полного графа приравнивается к минимальному весу пути, соединяющему вершины в исходном графе.
\end{definition}

\begin{proposition}
    $d$ -- есть метрика.
\end{proposition}

\begin{proof}
    $d$ удовлетворяет неравенству треугольника, остальное очевидно.
\end{proof}

\begin{proposition}
    Построение взвешенного графа возможно за полином. время от кол-ва вершин.
\end{proposition}

\begin{proof}\
    Единственный сложный шаг построения - вычисление всех кратчайших путей. Это возможно сделать, запустив алг. дейкстры от каждой вершины.
\end{proof}

\subsection{Корректность метрической версии}

Покажем явно, что полученный граф эквивалентен предыдущему с точки зрения веса минимального дерева Штейнера:
\begin{proof}

\

    $(\Rightarrow)$ Пусть $T$ - д.Ш. (здесь и далее "дерево Штейнера") в $G$, рассмотрим его образ при построении: рассмотрим подграф $T'$ в $G'$, получим его сохранив каждое ребро из $T$ ($d(u,v)\leq \omega({u, v})$), т.о. мин. вес д.Ш. в $G'$ не больше $\omega(T)$, т.е. веса мин. д.Ш. в $G$.
    
    $(\Leftarrow)$ Рассмотрим мин. д.Ш в $G'$, пусть это $T'$, каждое его ребро "развернём" \ в исходном графе, получим $T_r \subset G$. $T_r$ содержит все терминалы по построению, но в нём могли появиться циклы. Удалим их $(T_r \rightarrow T)$ и получим, что $\omega(T) \leq \omega(T')$.
\end{proof}
