\section{$\mathbf{NP}$-полнота}

Докажем, что задача поиска дерева Штейнера ($\mathsf{STEINERTREE}$) является $\mathbf{NP}$-полной, используя свведение из задачи о нахождении вершинного покрытия.

\begin{definition}
    \textbf{Задача} $\mathsf{STEINERTREE}$: дан взвешенный граф $G'=(V',E')$, множество терминалов $V_0'\subset V$ и число $K$. Также имеются веса на рёбрах $w': E' \rightarrow \mathbb{R}^{+}$. Требуется определить, существует ли дерево, содержащее все вершины $V_0$, веса не больше $K$.
\end{definition}

\begin{note}
    $\mathsf{STEINERTREE}\in \mathbf{NP}$: по предъявленному дереву можно за полиномиальное время
    проверить, что оно связно, содержит все терминалы и его суммарный вес $\le K$.
\end{note}

\begin{definition}
    \textbf{Задача} $\mathsf{VERTEXCOVER}$: дан граф $G = (V, E)$ и $k$ - количество вершин. Необходимо определить, существует ли в $G$ $k$ вершин, которые инцидентны всем рёбрам $E$.
\end{definition}

\begin{decision}
    Построим $f: G \rightarrow G'$ явно: для любой вершины $v$ из $V$ создадим нетерминальную вершину $v' \in G'$; для каждого ребра $e \in G$ создадим терминальную вершину $v_e' \in G'$ и соединим её рёбрами $g_{e1}'$ и $g_{e2}'$ веса 1 с образами инцидентных $e$ вершин. Кроме того, в $G'$ создадим нетерминальну вершину $n$ и соединим её рёбрами $g_n'$ веса 1 со всеми нетерминальными вершинами $G'$.
\end{decision}

\begin{proposition}
    Для графа $f(G)$ (где $G$ имеет вершинное покрытие мощности $k$) можно построить дерево Штейнера весом не больше, чем $\omega'(E') + k$.\
\end{proposition}

\begin{proof}\
    \begin{enumerate}
        \item Построим подграф $T'$ и покажем, что это дерево Штейнера: рассмотрим множество $X$, явл. верш. покрытием $G$, для каждой вершины $x \in X$ покрасим в красный её образ. Для каждой терминальной вершины $v_e'$ возьмём одно ребро, соед. её с красной вершиной, пусть оно приходит в вершину $u$ (возможно в силу существования вершинного покрытия $X$). Так же для каждой такой вершины $u$ возьмём ребро, соединяющее её с вершиной $n$.\
        Взяв корнем дерева вершину $n$, получим, что его глубина ровно 3 (вырожденный граф $G$ с нулём рёбер тривиален), т.к. терминальная вершина может быть соединина только с вершиной типа $v'$, а те, в свою очередь - только с вершиной $n$. Получили, что все терминалы доступны + нет циклов.
        \item По построению $T'$ очевидно, что $\omega'(T') \leq \omega'(E') + k$
    \end{enumerate}
\end{proof}

\begin{proposition}
    Если в графе $f(G)$, существует дерево Штейнера $T'$, т.ч. $|T'| \leq |E| + k$, то мы умеем явно предъявлять вершинное покрытие $G$ на $k$ вершинах\
\end{proposition}

\begin{proof}\
    \begin{enumerate}
        \item Найдём вершинное покрытие исходного графа: рассмотрим множество вершин $T'$, соединённых с терминалами. По построению они соответсвуют вершинам графа $G$ и все терминалы покрыты $\rightarrow$ они получены из вершин, на которых можно выбрать вершинное покрытие, пусть $X$
        \item Мы взяли все терминалы, т.е. $|T'| \geq |E|$, но при этом $|T'| \leq |E| + k$, т.е. оставшиеся связи с $n$ дополнили $|T'|$ не строго меньше, чем на k $\rightarrow |X| \leq k$
    \end{enumerate}
\end{proof}
