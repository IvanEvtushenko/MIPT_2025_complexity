\section{Обзор существующей литературы}

Задача поиска дерева Штейнера - есть одна из классических $\mathbf{NP}$-полных задач, последнее было доказано Ричардом Карпом в 1972 году~\cite{surs1}. Задача не имеет точного алгоритмического решение за полиномиальное время, если только не $\mathbf{P}=\mathbf{NP}$, поэтому естественное желание найти приближенный алгоритм.

Первый такой был предложен Такахаси и Мацуямой в 1980 году~\cite{surs2}. После Бырковой и др. в 2010 году был перестроен и реализован алгоритм, использующий метод релаксации направленного покомпонентного разреза исходной задачи, что позволило получить апроксимацию для restricted-варианта $ln(4)$, что даёт $ln(4) + \varepsilon$ для исходной задачи. 

Существует также жадный алгоритм, который был предложен Чи-Йен Чен в 2018 году~\cite{surs3}, который гарантирует решение с константой не хуже, чем $1.4295$.

Кроме того, в открытом доступе были обнаружены датасеты исходных графов для поиска деревьев Штейнера~\cite{surs4}, на которых был протестирован простейший алгоритм, дающий 2-приближение.